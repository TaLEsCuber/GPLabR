%!TEX program = xelatex
\documentclass[dvipsnames, svgnames,a4paper,11pt]{article}
% ----------------------------------------------------
%   中山大学物理与天文学院本科实验报告模板
%   作者:Huanyu Shi,2019级
%   知乎:https://www.zhihu.com/people/za-ran-zhu-fu-liu-xing
%   Github:https://github.com/huanyushi/SYSU-SPA-Labreport-Template
%   Last update : 2023.4.10
% ----------------------------------------------------

\input{Settings} % 导入模板的相关设置
\usepackage{lipsum}
\usepackage{enumitem}
\usepackage{tabularray}  %绘制表格时可以更加方便添加框线
\usepackage{xcolor} %添加更多文本颜色

\setlist[enumerate]{label=\textup{(\arabic*)}}



%---------------------------------------------------------------------
%	正文
%---------------------------------------------------------------------

\begin{document}


% \begin{table}
% 	\renewcommand\arraystretch{1.7}
% 	\begin{tabularx}{\textwidth}{
% 		|X|X|X
% 		|X|X|X|}
% 	\hline
% 	\multicolumn{2}{|c|}{预习报告}&\multicolumn{2}{|c|}{实验记录与分析}&\multicolumn{2}{|c|}{总成绩}\\
% 	\hline
% 	\LARGE30 & & \LARGE50 & & \LARGE80 & \\
% 	\hline
% 	\end{tabularx}
% \end{table}

\begin{table}
	\renewcommand\arraystretch{1.7}
	\begin{tabularx}{\textwidth}{
		|>{\centering}X|>{\centering}X|>{\centering}X
		|>{\centering}X|>{\centering}X|>{\centering\arraybackslash}X|}
	\hline
	\multicolumn{2}{|c|}{预习报告}&\multicolumn{2}{c|}{实验记录与分析}&\multicolumn{2}{c|}{总成绩}\\
	\hline
	\LARGE30 & & \LARGE50 & & \LARGE80 & \\
	\hline
	\end{tabularx}
\end{table}


\begin{table}
	\renewcommand\arraystretch{1.7}
	\begin{tabularx}{\textwidth}{|X|X|X|X|}
	\hline
	年级、专业:& 物理学 &组号:& 实验班2\\
	\hline
	姓名:& 戴鹏辉  & 学号: & 2344016 \\
	\hline
	日期:& 2024/03/25 & 教师签名:& \\
	\hline
	\end{tabularx}
\end{table}

\begin{center}
	\LARGE 光学像差实验I
\end{center}

\textbf{【实验报告注意事项】}
\begin{enumerate}[label=\arabic*., leftmargin=*]
	\item 实验报告由两部分组成:
		\begin{enumerate}[label=\arabic*), leftmargin=*]
			\item 预习报告:课前认真研读\underline{\textbf{实验讲义}},弄清实验原理;实验所需的仪器设备、用具及其使用、完成课前预习思考题;了解实验需要测量的物理量,并根据要求提前准备实验记录表格(可以参考实验报告模板,可以打印)。\textcolor{red}{\textbf{(30分)}}
			\item 实验记录与分析:认真、客观记录实验条件、实验过程中的现象以及数据。实验记录请用珠笔或者钢笔书写并签名(\textcolor{red}{\textbf{用铅笔记录的被认为无效}})。\textcolor{red}{\textbf{保持原始记录,包括写错删除部分,如因误记需要修改记录,必须按规范修改。}}(不得手记的值输入到电脑打印);离开前请实验教师检查记录并签名。\textcolor{red}{\textbf{(50分)}}
		\end{enumerate}
	
	\item \textcolor{red}{\textbf{本实验报告可提前打印出来,当场记录分析完成交给带实验的老师,课后无需再提交。若当场完成不了,则请课后完成,再扫描并通过seelight提交。}}
	
	\textcolor{red}{\textbf{注意:本文档已留出填写空间,若填写空间不够的话请提前规划留白,做到报告的美观}}
	\item 注意事项:
		\begin{enumerate}[label=\arabic*), leftmargin=*]
			\item 实验中\textcolor{red}{\textbf{避免激光器伤到眼睛}}
			\item 避免用手直接接触镜片的光学面
			\item 安装镜片时需在光学平台上尽量靠近台面的高度操作,以免失手跌落摔碎镜片
			\item 实验平台配件所用固定螺钉需拧紧,以免镜架晃动;但不可过紧,以免损坏
			\item 实验前需按仪器清单检查光学元件是否齐全,\textcolor{red}{\textbf{实验结束后按照顺序放回元件盒}}
			
		\end{enumerate}
\end{enumerate}


\clearpage
\tableofcontents
\clearpage

\setcounter{section}{0}
\section{光学像差实验I \quad\heiti 预习报告}
	
\subsection{实验目的}
	\begin{enumerate}
		\item 了解七种几何像差产生的原理、基本规律;
		\item 了解各种像差对光学成像质量的影响;
		\item 掌握慧差、色差产生的原理及其测量表征; 
		\item 掌握光学系统的基本调试方法;
		
		
		
	\end{enumerate}

\subsection{仪器用具}
% \begin{table}[htbp]
% 	\centering
% 	\renewcommand\arraystretch{1.6}
% 	% \setlength{\tabcolsep}{10mm}
% 	\begin{tabular}{p{0.05\textwidth}|p{0.20\textwidth}|p{0.05\textwidth}|p{0.5\textwidth}}
% 	\hline
% 	编号& 仪器用具名称 & 数量 &  主要参数(型号,测量范围,测量精度等) \\
% 	\hline
% 	1  & 防震光学平台 & 1  & ~  \\
% 	2  & 氦氖激光器  & 1  & ~  \\
% 	3  & 扩束透镜   & 1  & ~  \\
% 	4  & 分束器    & 1  & ~  \\
% 	5  & 反射镜    & 3  & ~  \\
% 	6  & 全息干版   & 1  & ~  \\
% 	7  & 显影液    & 1  & ~  \\
% 	8  & 定影液    & 1  & ~  \\
% 	9  & 暗房设备   & 1  & ~  \\
% 	\hline
% \end{tabular}
% \end{table}

\begin{table}[htbp]
    \centering
    \begin{tabular}{|c|c|c|c|}
        \hline
        产品编号 & 产品名称 & 规格 & 数量 \\
        \hline
        1 & 激光光源 & $\lambda =632.6nm$ & 1 \\
        2 & 激光器夹持器 & 3维调整 & 1 \\
        3 & 显微物镜 & 10×/0.25 & 1 \\
        4 & 针孔 & Ø100um或Ø50um & 1 \\
        5 & 五维调整机构 & 装配显微物镜和针孔用 & 1 \\
        6 & 衰减片1 & 0.0001(衰减系数),装在镜框中 & 1 \\
        7 & 衰减片2 & 0.01(衰减系数),装在镜框中 & 1 \\
        8 & 双凸透镜1 & 焦距$f=300mm$,装在透镜/反射镜座中 & 1 \\
        9 & 平凸透镜2 & 焦距$f=100mm$,装在透镜/反射镜座中 & 1 \\
        10 & 平凸透镜3 & 焦距$f=150mm$,装在透镜/反射镜座中 & 1 \\
        11 & 白屏 & SZ-13,一面白屏,一面带坐标纸 & 1 \\
        12 & 成像相机 & 大恒的MER-130-30UM或元启智能的REV-13AMU2C & 1 \\
        13 & 数据线 & 图像采集数据线 & 1 \\
        14 & 计算机 & 台式或笔记本,安装有成像相机图像采集软件 & 1 \\
        15 & 光学导轨 & 长度1米,带刻度 & 1 \\
        16 & 二维平移台 & 行程>10mm & 4 \\
        17 & 平移滑块 & & 8 \\
        18 & 支杆 & 50mm长和75mm长 & 3+5 \\
        19 & 磁性表座 & & 4 \\
        20 & 旋转调整台 & 可调角度>±5° & 1 \\
        21 & 白光灯 & GY-6型,亮度可调,即溴钨灯 & 1 \\
        22 & 滤光片1 & 透光波长:435nm & 1 \\
        23 & 滤光片2 & 透光波长:630nm & 1 \\
        \hline
    \end{tabular}
\end{table}



\subsection{原理概述}
(15分)(概述色差和慧差产生的原理)(请用自己的语言描述,勿大幅copy讲义等)(填写空间不够的话请提前规划留白,做到报告的美观)


像差理论是光学设计求解光学系统初始结构的理论基础。在建立起理想光学系统后,将实际光学系统所成的像偏离理想光学系统的误差称为几何像差,简称像差。光学设计者将几何像差分为七种,即球差、慧差、像散、场曲、畸变、位置色差和倍率色差。产生像差的原因有三点:
	\begin{enumerate}
		\item 光线计算公式的非线性;
		\item 物面为平面,折(反)射面为球面(曲面),成像面为曲面;
		\item 不同颜色(波长)的光在折射介质中的折射率不同。
	\end{enumerate}
	
	\begin{enumerate}
		\item 色差(Chromatic Aberration):色差是由于不同波长(颜色)的光线在透镜或系统中的折射率不同而引起的。这导致不同波长的光线会聚或发散到不同的焦点位置,从而造成成像时不同波长的光线无法同时聚焦于同一平面上,使得图像产生色彩偏差。色差分为两种:
			\begin{enumerate}
				\item 焦距色差(Longitudinal Chromatic Aberration):不同波长的光线在透镜中折射后,聚焦于不同的位置,导致成像平面不同。
				\item 横向色差(Lateral Chromatic Aberration):不同波长的光线在透镜中折射后,沿不同的轴向散开,使得不同波长的光线成像位置有所偏移。
			\end{enumerate}

		
		\item 慧差(Spherical Aberration):慧差是由于透镜或反射面的形状不是理想的球面而引起的。理想的球面透镜或反射面能使所有入射光线汇聚于一个焦点,但非理想的球面会导致不同位置的光线汇聚于不同的焦点,产生像差。慧差会使成像的图像产生模糊和失真。
	\end{enumerate}

	

\subsection{实验前思考题}
	\begin{question}
		慧差与孔径、视场的关系?
	\end{question}
		
		
			
		
		
		
		
		

	\begin{question}
		产生色差原因?列举几种消色差的方法
	\end{question}




	\begin{question}
		针孔滤波的工作原理
	\end{question}

\clearpage
\begin{table}
	\renewcommand\arraystretch{1.7}
	\centering
	\begin{tabularx}{\textwidth}{|X|X|X|X|}
	\hline
	专业:& 物理学 &年级:& 2022级 \\
	\hline
	姓名:& 戴鹏辉 & 学号:& 22344016 \\
	\hline
	室温:&  & 实验地点: &  \\
	\hline
	学生签名:& & 评分: &\\
	\hline
	日期:&  & 教师签名:&\\
	\hline
	\end{tabularx}
\end{table}

\section{光学像差实验I \quad\heiti 实验记录}
\subsection{实验内容、步骤、结果及讨论}\textcolor{ForestGreen}{(按照实验顺序依次}\textcolor{red}{简要记录}\textcolor{ForestGreen}{实验内容及步骤,)(空间不够,可自行加页)}\\
\textcolor{red}{
(注意: \\
除了记录实验内容、步骤、参数外,还应记录:\\
按比例绘制操作中实际摆放的实验光路(各元件间距离可通过直尺测量)\\
记录光路中物光和参考光的光程差\\
记录物光和参考光光强比\\
记录是否可观察到再现图像\\
)
}


		

%\subsection{实验数据记录}



%\subsection{原始数据记录}

\clearpage

\newpage

\null

\newpage

\null






\newpage

\subsection{实验过程中遇到的问题记录}

	\begin{enumerate}
		\item 
		\item 
		\item 
	\end{enumerate}
\null




% \begin{table}
% 	\renewcommand\arraystretch{1.7}
% 	\begin{tabularx}{\textwidth}{|X|X|X|X|}
% 	\hline

% 	专业:& 物理学 &年级:& 2022级\\
% 	\hline
% 	姓名: & 戴鹏辉 & 学号:& 22344016\\
% 	\hline
%     日期:& 2024/xx/xx & 评分: &\\
% 	\hline
% 	\end{tabularx}
% \end{table}

% \section{全息照相实验 \quad\heiti 分析与讨论}

% \subsection{实验数据分析}

% 	\subsubsection{实验一 测量光栅常数}
	
		
% 	\subsubsection{实验二 测定未知光波波长及角色散率D}
			

			
% \subsection{实验后思考题}



% \begin{question}
% 	检索文献,列举三种测量光波波长的方法,给出参考文献列表。%\lipsum[20]
% \end{question}
	

	



\end{document}
